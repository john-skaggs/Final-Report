\chapter{Introduction}\label{sec:1}


\section{Project Description \& Objectives}\label{sec:1.1}

Bridging the Gap Engineering has been pre-qualified by the West Virginia Division of Highways (WVDOH) to perform the engineering design work for a new 125 foot span bridge responsible for carrying vehicular traffic on Begley Road, located in Cabell County, West Virginia, across Bryce Creek. The bridge over Bryce Creek is intended to be two 12-foot-wide lanes with a 3-foot-wide shoulder on the side of each lane, it is intended to accommodate an average daily traffic (ADT) of 5000 vehicles per day, and must also allow for a 12\degree \hspace{0.05cm} skew.

Prior to generating the final design, it was deemed prudent to analyze three different methodologies common to bridge design in order to determine the most feasible alternative in the interest of better serving the client and the travelling public.  The following report details the advantages and disadvantages associated with the use of different girder materials for the bridge's superstructure and the reasoning that led to our decision as to which material best suits our needs.

\section{Design Codes \& Standards}\label{sec:1.2}

The bridge over Bryce Creek must follow the $7^{th}$ Edition of AASHTO LRFD Bridge Design Specifications~\cite{aashto7}. In order to determine truck traffic Bridging the Gap Engineering was directed to use the estimation methods provided in~\cite{aashto7}. In addition to the specifications detailed in~\cite{aashto7}, all federal and state guidelines that are applicable must also be followed.  The relevant state codes can be found in the following publications published by the WVDOH,~\cite{bridgedesignman},~\cite{detail1},~\cite{detail2},~\cite{detail3}.  In addition, Bridging the Gap Engineering  was directed to conservatively neglect hydraulic opening requirements, employ the optional live load deflection limit of $\frac{L}{800}$, and design the deck using concrete that meets a $28$-day compressive strength of 4.0 ksi.

\section{Scope of Work}\label{sec:1.3}
Our group was tasked with the design of the bridge deck and the accompanying shear layouts, selection and design of girders, both moment and shear along with necessary serviceability accommodations, proportioning lateral bracing if called for, and the preparation of bridge plans, cost estimates, and construction estimates associated with the aforementioned responsibilities.

\section{Description of Alternative Designs}\label{sec:1.4}
Three design alternatives were considered for the Begley Road bridge.  These alternatives, listed in the order they appear, were the use of Steel Plate Girders for the construction of the bridge's superstructure, Concrete I-Beams for the construction of the bridge's superstructure, or Steel Rolled Beams for the construction of the bridge's superstructure.  Ultimately each design was viable, so the final selection was based upon economic feasibility.
