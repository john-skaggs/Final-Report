\chapter{Alternative Design \#3- Steel Rolled Beam}


\section{General Description of Design Alternative}
Steel rolled beams could provide a feasible, though not ideal, alternative for the construction of a single span, 125 foot long bridge.  These girders can support the load and deflection requirements of the bridge. They would bear onto the concrete abutments through the use of elastomeric bearings for each beam. Lateral bracing will be included in the final superstructure design.


\section{Assumptions Made}
Manufacturing and shipping limitations on length of rolled girders of this size indicate that 100 foot and 25 foot sections would need to be utilized and spliced and bolted together into the final structure.  This length is assumed to be in the capability of the steel manufacturer, though could not be confirmed due to the lack of communication of the steel manufacturer.


\section{Design Details}
Rolled girders are constructed by a steel manufacturer from one singular cross section and extruded to length from its rolled billet. The advantage of this form of steel in the use of construction is in its ready supply and reduced manufacturing costs in comparison to manufactured beams made from plates.

The cross section of the bridge, for the rolled girder alternative, consists of 4 girders, spaced 9.0 feet apart, measured center-to-center. The beams will be seated on 18 inch \(\times\) 20 inch \(\times\) 5.875 inch elastomeric bearing pads with steel plate reinforcement~\ref{misc:espan}. A total of 8 elastomeric bearings will be utilized in this design. The sections of rolled beam will be spliced together at 100 foot, to accommodate the manufacturing and shipping limits.

The design of the bridge deck will be the same as the plate girder alternative; therefore, the deck overhang is 34 inches, meeting the minimum overhang stated in Section 3.2.1.1 of~\cite{bridgedesignman}. The deck will have an overall thickness 8.5 inches, with 0.25 inches being for the integral wearing surface. The deck will contain two layers of \#5 reinforcement, one layer will run parallel to the skew of the bridge and the other will run perpendicular to traffic. The layer of reinforcement running parallel to traffic will contain 22 \#5 bars and be spaced at 18 inches. The layer of reinforcement running perpendicular to traffic will contain 83 \#5 bars and be spaced at 18 inches. The deck surface will contain 2 TL-3 Type F (32 inch) barrier, according to Section 3.2.2 of~\cite{bridgedesignman}.

Ultimately, this results in each girder needing to resist a live load of 1.713 \(\frac{\textrm{kips}}{\textrm{foot}}\) and a maximum moment of 10215.625 foot-kips of moment per beam. In order for these girders to hold the weight of the bridge deck as well as their own self weight, the bending moment imposed by the dead and live loads and meet the tolerance for live load deflection, a W40\(\times\)593 was calculated to be the minimum size necessary to withstand these parameters.

This was determined through the calculation of the minimum I\(_{x}\) necessary as imposed by the live load per foot multiplied by 1.75, and within the confines of the \(\frac{L}{800}\) maximum deflection parameter.  The calculation performed made use of the following formula:
\begin{equation*}
I_{x}=\dfrac{5w_{L}L^{4}}{384E\Delta_{LL,\max}}
\end{equation*}
This resulted in a minimum I\(_{x}\) of 14420 inch\(^{4}\).  There are several rolled shapes the fit this requirement, the first and lightest being the W40\(\times\)199.  Checking the  Z\(_{x}\) tables, however, the $\phi M_{n}$ of this shape was far too low, being only 3260 foot-kips, when 10215.625 foot-kips is required. Moving up the line of W-shapes with increasing I$_{x}$ values and checking their moment capacities, ultimately the W40x593 is the lightest beam that can withstand both the Ix and moment requirement.

\section{Quantity Estimates}
At 593 $\frac{\textrm{lb}_{f}}{\textrm{foot}}$ of steel and four beams, the total weight of steel necessary would be 296,500 pounds, or 148.25 tons. This does not include the necessary plates and bolts to connect the lengths at the 100 foot mark imposed by the shipping constraint.


\section{Scheduling Estimates}
According to provided information, we estimate that the Rolled Beam Alternative will take approximately 5 weeks to complete. This estimate does not include shipping delays or construction delays due to weather. This was calculated assuming an eight-hour day and five-day work week.

\section{Cost Estimates}
The cost analysis for the Rolled Steel Girder Alternative excludes any project features that all three alternatives shared. In other words, the cost analysis only considered costs that were specific to the Rolled Steel Girder and were the result of characteristics unique to the alternative. Features excluded contain, but are not limited to, the concrete deck, deck formwork, parapet walls, and labor for these features.

An online crane calculator was used to determine the minimum sized crane necessary to lift the Concrete Type V I Beam into place using the weight of the load and the radius that the crane has to carry the load outward as parameters. It was determined that the minimum crane size required to place the steel plate girders is a 160 ton crane. Due to a lack of reliable contacts, the shipping was estimated to be 10\% of the girders cost. This rate is used in each alternative's cost estimate, however the actual cost of shipping will vary for each alternative.

A 5\% lump sum was added to account for sheer studs and any other miscellaneous features that might have not been taken into consideration. Markups were assumed and added to the total project cost. These include a running percent of 15\% on job office overhead (Primary Contractor), 15\% on home office overhead (Primary Contractor), 10\% on Profit (Primary Contractor), 2\% on bond (Primary Contractor), and 6\% on taxes (Project). Sub-contractors also have a 25\% running percent that includes job office overhead and profit. The direct cost of this alternative came out to roughly \$790,000 with a project cost of \$1,200,000. This was found to be more than double what the cost of the lowest alternative is projected to cost, which results from a weight that is twice as heavy as the steel plate girder alternative.




\begin{table}[H]
\centering
\caption{Rolled Steel Beam Alternative Cost}
\vspace{0.5cm}
\scalebox{0.85}{
\begin{tabular}{p{5.0cm}rrrrr}
\toprule
\midrule
\textbf{Description}                                                                                               &  \textbf{Quantity} & \textbf{UOM}   & \textbf{Contractor} & \textbf{Direct Cost} &\textbf{Project Cost} \\   \midrule
Rolled Steel Beam with Installation                                                                                  &  52     & TON & Prime  & \$669,766  &  \$993,829     \\ \midrule
Shipping for Rolled Beam                                                                                           &  1      &  LS & Sub    & \$127,284  &  \$68,624   \\ \midrule
Rolled Steel Girder Misc. (Bolts, Sheer Studs, etc.)                                                               &  1      & LS  &  Prime & \$50,725   & \$34,185\\ \midrule
M 164 Type 3 Bolts, \(\frac{7}{8}\) inch diameter, includes nut \& washer.                                                                 &  1,312  & EA  &  Prime & \$8,193    & \$12,157\\ \midrule
Top Flange: Steel plate, structural, for connections \& stiffeners, \(\frac{1}{2}\) inch T, shop fabricated, includes shop primer      &  4898   &  SI &  Prime &  \$973     & \$1,444\\ \midrule
Bottom Flange: Steel plate, structural, for connections \& stiffeners, 1 inch T, shop fabricated, includes shop primer     &  9174   & SI  &  Prime &  \$1,768   & \$2,623\\ \midrule
Web: Steel plate, structural, for connections \& stiffeners, \(\frac{1}{2}\) inch T, shop fabricated, includes shop primer             &  12,240 & SI  & Prime & \$2,432     &  \$3,609\\ \midrule
Splice Connection Installation, Crane- 40 Ton                                                                      &  4      &  EA & Prime &  \$3,168    & \$4,701\\ \midrule
Total                                                                                                              &        &    &      & \$790,000   & \$1,200,000  \\ \midrule
\bottomrule
\end{tabular}}
\end{table}




\section{Sustainability-Related Issues}
During the preliminary design performed for the alternatives phase, there were no sustainability issues observed with this alternative.
