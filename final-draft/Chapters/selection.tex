\chapter{Selection of Design Alternative}

\section{Selection Methodology}
The three alternatives included in this report were initially compared on a basis of cost, shipping limitations, scheduling, manufacturing availability, and capacity requirements. However, shipping limitations immediately disadvantaged one of the alternatives due to the single span requirement and the shipping limit of 100 feet. Manufacturing availability did not prove to be as large an issue as initially expected. Upon research, most of the components of the alternative could be fabricated/manufactured without issue. Scheduling did not end up playing a large factor in the alternative selection because the construction time estimates were relatively close. Overall alternative cost was the final determining factor in the two alternatives that met the shipping criteria.

\section{Comparison of Design Alternatives}


\begin{table}[H]
\centering
\caption{Cost Comparison of Alternatives}\label{tab:costcomp}
\vspace{0.5cm}
\begin{tabular}{p{4.9cm}rrr}\toprule\midrule
\textbf{Alternative}                 & \textbf{Directed Material Cost}  & \textbf{Project Cost} & \textbf{Direct Cost} \\ \midrule
\emph{Steel Plate Girders}           & \$215,000                        & \$445,000             & \$295,000             \\ \midrule
\emph{AASHTO Type V Concrete I-Beam} & \$220,000                        & \$500,000             & \$330,000              \\ \midrule
\emph{Steam Rolled Beam}             & \$230,000                        & \$510,000             & \$340,000               \\ \bottomrule
\end{tabular}
\end{table}

Table~\ref{tab:costcomp} displays the major cost points for each alternative and summarizes the overall costs for comparison. Cost was a determining factor in selecting the two feasible alternatives that remained after vetting each alternative with the shipping requirements.

The Steel Plate Girder alternative was the most appealing alternative from the beginning of the preliminary design. This was mostly due to the flexibility of the girder design and fabrication. The ability to splice the 100 foot and 25 foot sections together was a critical feature given the 125 foot single span parameters. This alternative also proved well during the cost analysis, as being the cheapest and most feasible alternative.

The AASHTO Type V Concrete I-Beam appeared to be an alternative with quicker assembly and lower labor hours; however, this alternative was not compatible with the shipping length limitation of 100 feet. Given that the bridge is a single span, there was no feasible way to provide a 125 foot span of these precast, prestressed girders.

The Steel Rolled Beam alternative seemed appealing initially, in hopes that the cost would be less since it was a readily manufactured rolled W-shape. This alternative would have had to also been spliced at 100 feet to accommodate the shipping requirements. This added labor costs and the cost of the connection itself. Overall, this alternative ended up being the most expensive, and was not feasible.

\section{Selected Alternative}
The Steel Plate Girder is the alternative that was selected moving forward with the remainder of the bridge design and analysis. The plate girder provided the team with solutions for design flexibility, feasible costs, and ease of shipping. The selection of the plate girder hinged on two main issues with the other two alternatives: the shipping length disqualified the AASHTO Type V Concrete I-Beams, and the cost of the Rolled Beam alternative in direct comparison to the Plate Girder was too high.

%\section{Remaining Design Tasks}
%The following design tasks remain for the team moving forward:
%\begin{itemize}
%\item Refined Shear \& Moment Analysis
%\item Lateral Bracing Design
%\item Camber Design
%\item Approach Slab Design
%\item Shear Stud Design
%\item Steel Reinforcement Analysis
%\item Connection Design Refinement \& Analysis
%\item Girder optimization
%\end{itemize}