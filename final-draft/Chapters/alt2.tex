\chapter{Alternative Design \#2- Concrete I-Beam}

\section{General Description of Alternative Design}
For the concrete alternative, Bridging the Gap Engineering researched a number of reinforced concrete girder designs. Using the calculated load combinations and requested span length along with Tables 2.4.2 and 2.4.2.3 found in~\cite{bridgedesignman} Bridging the Gap Engineering determined that the AASHTO Type V I-Beam was the appropriate beam for the given scenario. The AASHTO Type V girder measures 63 inches deep and 42 inches wide at the top flange; note that the appropriate cross section can be found in Appendix~\ref{cad:concibeamcross}.  Using the aforementioned tables, it was determined that the concrete alternative will require 4 girders spaced at 8 feet apart to reach the 125 foot span length.

\section{Assumptions Made}

As issues with using a concrete alternative began to arise early on it was quickly assumed that this alternative would not be feasible. Due to the same shipping restriction from Section~\ref{sec:2}, our concrete girder alternative was disproportionately negatively impacted. This restriction prevents shipping a 125 foot beam, however unlike in the previous case, it would not be feasible to somehow splice together shorter girders to achieve the desired length. Additionally, multiple companies were contacted regarding the possible fabrication of the beam. Each stated that while making a beam that length can be done, the integrity of such a beam is risked in transit and was therefore highly advised against.

\section{Design Details}

The concrete girders were designed to withstand its own dead loads as well as the live loads that would be be generated by the HL-93 design truck. Table~\ref{tab:concprops} below contains the dimensions necessary for a AASHTO Type V I-Beam to resist the load scenario.

\begin{table}[H]
\centering
\caption{Properties of AASHTO Type V I-Beam}\label{tab:concprops}
\begin{tabular}{lr}\toprule\midrule
\textbf{Beam Properties:}                      &        \\\midrule
\emph{Area (in\(^{2}\))}                       & 1013    \\
\emph{I (in\(^{4}\))}                          & 521180   \\
\emph{Deck Thickness (in)}                     & 9         \\
\emph{Length (ft)}                             & 125        \\
\emph{Spacing (ft)}                            & 8           \\
\emph{y\({\textrm{bottom}}\)}                  & 31.96        \\
\emph{skew (\degree)}                          & 12            \\
\emph{\(\gamma (\frac{lb}{\textrm{ft}^{3}})\)} & 150\\\bottomrule
\end{tabular}
\end{table}

\section{Quantinty Estimates}

The following quantities were estimated for the AASHTO Type V Concrete I-beam design, using both raw material weights and linear footage quantities. Table~\ref{tab:concquant3} displays the total quantities for the concrete. Table~\ref{tab:babquant3} displays the quantities for the bearings.


\begin{table}[H]
\centering
\caption{Concrete Quantities}\label{tab:concquant3}
\scalebox{0.5}{
\begin{tabular}{p{4.4cm}rrrrrrrrrrr}\toprule\midrule
\textbf{Part}                                                              & \textbf{\makecell{Length\\ (ft)}}  & \textbf{\makecell{Width\\(ft)}} & \textbf{\makecell{Thickness\\(ft)}} & \textbf{\makecell{Area\\(in\(^{2}\))}} & \textbf{\makecell{Area\\(ft\(^{2}\))}} & \textbf{\makecell{Volume\\(ft$^{3}$)}} & \textbf{\makecell{Number\\Required}} & \textbf{\makecell{Weight\\(lb)}} & \textbf{\makecell{Weight per\\Part}} & \textbf{\makecell{Total Weight\\(lbs)}} & \textbf{\makecell{LF of Concrete\\Girders}} \\\midrule
\emph{AASHTO I-Beam -Type V}                                               & 125                                &                                 &                                     & 1013.00                                & 7.03                                   &                                        & 4.00                                 & 1055.00                          & 4220.00                              & 16880.00                                & 500.00                                       \\
\emph{Concrete Deck}                                                       & 125                                & 30.00                           & 0.75                                &                                        & 22.50                                  & 2812.50                                & 1.0                                  &                                  & 407812.50                            & 407812.50                               & 125.00                                        \\
\emph{Type V Median Parapet Class B Concrete (assuming rectangular shape)} & 125                                & 0.79                            & 2.67                                &                                        & 2.11                                   & 263.9219                               & 2.0                                  &                                  & 38268.67                             & 76537.34                                & 250.00                                         \\\midrule
Total:                                                                     &                                    &                                 &                                     &                                        &                                        &                                        &                                      &                                  &                                      & 501229.84                               & 875.00                                          \\\bottomrule
\end{tabular}}
\end{table}



\begin{table}[H]
\centering
\caption{Bearing Quantities}\label{tab:babquant3}
\vspace{0.3cm}
\begin{tabular}{p{3.0cm}rrrrr}\toprule\midrule
\textbf{Part}                   & \textbf{Length (in)} & \textbf{Width (in)} & \textbf{Thickness (in)} & \textbf{Number Required}\\\midrule
\emph{Elastomeric Bearing Pads} & 20                   & 18                  & 5.875                   & 8 \\\bottomrule
\end{tabular}
\end{table}

\section{Scheduling Estimates}
According to provided information, it is estimated that the AASHTO Type V Concrete I-Beam Alternative would take approximately 7 work weeks to complete. This estimate does not include any delays related to manufacturing, shipping, or construction issues due to weather. This estimate was calculated assuming 8 hour work days and 5 day work weeks.


\section{Cost Estimates}

The cost analysis for the Concrete Type V I-Beam Alternative excludes any project features that all three alternatives shared. In other words, the cost analysis only considered costs that were specific to the Concrete Type V I-Beam and were the result of characteristics unique to the alternative. Features excluded contain but are not limited to the concrete deck, deck formwork, parapet walls, and labor for these features.

An online crane calculator was used to determine the minimum sized crane necessary to lift the Concrete Type V I-Beam into place using the weight of the load and the radius that the crane has to carry the load outward as parameters. It was determined that the minimum crane size required to place the steel plate girders is a 400 ton crane. Due to a lack of reliable contacts, the shipping was estimated to be 10\% of the girders cost. This rate is used in each alternative's cost estimate, however the actual cost of shipping will vary for each alternative.

A 5\% lump sum was added to account for sheer studs and any other miscellaneous features that might have not been taken into consideration. Markups were assumed and added to the total project cost. These include a running percent of 15\% on job office overhead (Primary Contractor), 10\% on job office overhead (Primary Contractor), 10\% on Profit (Primary Contractor), 2\% on bond (Primary Contractor), and 6\% on taxes (Project). Sub-contractors also have a 25\% running percent that includes job office overhead and profit. The direct cost of this alternative came out to roughly \$330,000 with a project cost of \$500,000. This was found to be approximately 12\% higher than the lowest cost alternative.

\begin{table}[H]
\centering
\caption{Concrete Type V I-Beam Alternative Cost}
\vspace{0.5cm}
\begin{tabular}{p{3.5cm}rllrr}\toprule\midrule
\textbf{Description}                                 & \textbf{Quantity} & \textbf{UOM}  & \textbf{Contractor} & \textbf{Direct Cost} &\textbf{Project Cost} \\\midrule
Prestressed Concrete Type V I-Beam with Installation & 263.75            & TON           & Prime               & \$290,158            & \$40,440              \\\midrule
Shipping for \\Concrete I-Beam                       & 1                 & LS            & Sub                 & \$37,110             & \$68,883               \\\midrule
Total:                                               &                   &               &                     & \$330,000            & \$500,000               \\\bottomrule
\end{tabular}
\end{table}


\section{Sustainability-Related Issues}
During the preliminary design performed for the alternatives phase, there were no sustainability issues observed with this alternative.